\documentclass{beamer}

\usepackage[utf8]{inputenc}
\usepackage{amsmath,amsfonts,amsthm, amssymb, mathrsfs, yfonts} % Math packages
\usepackage{cancel}

\newcommand{\FF}{\mathbb{F}}
\newcommand{\A}{\mathcal{A}}
\newcommand{\C}{\mathcal{C}}

\newtheorem{thm}{Theorem}
\newtheorem{pf}{Proof}
\newtheorem{rmk}{Remark}
\newtheorem{defn}{Definition}


\usetheme{Copenhagen}
\title{Slide show test}
\author{Petar Hlad Colic}
\institute{Universitat Polit\`ecnica de Catalunya}
\date{May 2018}

\begin{document}
    \frame{\titlepage}
    
    \section{Preliminaries on LRC codes}

\begin{frame}{Notation}
    \begin{itemize}

	    \item $\C$ denotes a code over the finite field $\FF_q$.
	    
	    \item The triple of parameters $(n,k,r)$ refers to a code of:
	    \begin{itemize}
	        \item length $n$
	        \item cardinality $q^k$
	        \item locality $r$
	    \end{itemize}
	    
	    \item $[n] := \{ 1, \dots , n\}$
	    
	    \item A \textit{restriction} $\C_I$ of the code $\C$ to a subset of coordinates $I \subset [n]$ is the code obtained by removing from each vector the coordinates outside $I$.
    \end{itemize}
\end{frame}

\subsection{Definition and properties of LRC}

\begin{frame}{Definition of LRC Codes}
    Given $a \in \FF_q$ consider the sets of codewords of $\C$ with fixed value $a$ at the symbol $x_i$:

    $$ \C (i,a) = \{ x \in \C : x_i = a \}, \quad i \in [n] $$
    
    \begin{defn}
        A code $\C$ of length $n$ has \textbf{locality r} if $\ \forall i \in [n ] $ there exists a subset $I_i \subset [n] \setminus i, \quad \vert I_i \vert \leq r$ such that the restrictions of the sets $\C (i,a)$ to the coordinates in $I_i$ for different $a$ are disjoint:
        $$\C_{I_i} (i,a) \cap \C_{I_i} (i,a') = \emptyset, \quad a \neq a'.$$
    \end{defn}    
\end{frame}

\note{
Given $x \in \C$, $i \in [n]$, there exists a subset $I_i \subset [n] \setminus i$ of at most $r$ coordinates such that the restriction of $\C$ to $I_i$ enables to find the value of $x_i$.
    $I_i$ is called a \textbf{recovering set} for the symbol $x_i$.
    }
    
\begin{frame}{Bounds on rate and min. distance}
	Let $\C$ be an $(n,k,r)$ LRC code of cardinality $q^k$ over an alphabet of size $q$. Then:
    \begin{thm}[Upper bound on the rate]
        The rate of $\C$ satisfies
        $$ \frac{k}{n} \leq \frac{r}{r+1} $$
    \end{thm}
    
    \begin{thm}[Generalization of Singleton bound]
        The minimum distance of $\C$ satisfies
            $$d \leq n - k - \left\lceil \frac{k}{r} \right\rceil + 2$$
        A code that achieves the bound with equality will be called an \textbf{optimal LRC code}.
    \end{thm}
\end{frame}
    
    \begin{frame}
        \frametitle{Construction of LRC codes}
        We want to construct a linear $(n,k,r)$-LRC code. Assume $r \vert k$ and $(r+1) \vert n$.
        
        We need:
        
        \begin{itemize}
            \item $A_1 , \dots , A_{\frac{n}{r+1}}$ disjoint subsets of the field $\FF_q$, s.t. $\vert A_i \vert = r+1$
            \item $g(x) \in \FF_q[x]$ a polynomial s.t.
            \begin{enumerate}
                \item $deg(g) = r+1$
                \item $g$ is constant on each set $A_i$: $g(\alpha) = g(\beta)$ for $\alpha, \beta \in A_i$
            \end{enumerate}
        \end{itemize}                
        We will call $g$ a good polynomial.
        
    \end{frame}
    
    \begin{frame}{Construction of LRC codes}
        Let $A= \bigcup_{i=1}^{\frac{n}{r+1}} A_i \subset \FF_q$, $\vert A \vert = n$.
        
        We write now message vectors $a \in \FF_q^k$ as $r \times \frac{k}{r}$ matrices.
        
        $$ a = \;
        \begin{pmatrix}
            a_{0,0} & a_{0,1} & \cdots & a_{0,\frac{k}{r}-1} \\
            a_{1,0} & a_{1,1} & \cdots & a_{1,\frac{k}{r}-1} \\
            \vdots  & \vdots  & \ddots & \vdots \\
            a_{r-1,0} & a_{r-1,1} & \cdots & a_{r-1,\frac{k}{r}-1}
        \end{pmatrix}
        $$
    \end{frame}
    
    \begin{frame}{Construction of LRC codes}
        \begin{block}{Encoding polynomial}
            Given the message vector $a \in \FF_q^k$, define the \textbf{encoding polynomial} as:
            $$ f_a(x) = \sum_{i=0}^{r-1} x^i \cdot f_i(x) $$
            where
            $$f_i(x) = \sum_{j=0}^{\frac{k}{r}-1} a_{ij} g(x)^j $$
        \end{block}
    \end{frame}
    \begin{frame}
        $$ f_a(x) =
        \begin{pmatrix}
            x^0 & \dots & x^{r-1}
        \end{pmatrix}
        \begin{pmatrix}
            a_{0,0} & \cdots & a_{0,\frac{k}{r}-1} \\
            \vdots  & \ddots & \vdots \\
            a_{r-1,0} & \cdots & a_{r-1,\frac{k}{r}-1}
        \end{pmatrix}
        \begin{pmatrix}
            g(x)^0 \\
            \vdots \\
            g(x)^{\frac{k}{r}-1}
        \end{pmatrix} =
        $$
        
        $$
        =
        \begin{pmatrix}
            x^0 & \dots & x^{r-1}
        \end{pmatrix}
        \begin{pmatrix}
            f_0(x) \\
            \vdots \\
            f_{r-1}(x)
        \end{pmatrix}
        $$
    \end{frame}
    
    \begin{frame}
        The codeword for $a \in \FF_q^k$ is found as the evaluation vector of $f_a$ at all the points of $A$.
        \begin{block}{LRC code}
            The $(n,k,r)$ LRC code $\C$ is defined as the set of $n$-dimensional vectors
            $$\C = \{ (f_a(\alpha), \alpha \in A) : a \in \FF_q^k \}$$
        \end{block}
    \end{frame}
    
\begin{frame}
    \begin{rmk}
        $$x \in A_i \Rightarrow \quad g(x) \mbox{ constant}$$
        $$\Rightarrow f_\ell(x) = \sum_{j=0}^{\frac{k}{r}-1} a_{\ell j} g(x)^j \mbox{ constant in } A_i$$
        $$\Rightarrow \text{deg}(f_a(x)) = \text{deg}(\sum_{j=0}^{r-1} x^j \cdot f_j(x)) \leq r-1 \mbox{ in } A_i$$
    \end{rmk}
\end{frame}

\begin{frame}{Recovery of the erased symbol}
    Suppose erased symbol: $\alpha \in A_j$.
    
    Let $\left( c_{\beta}, \ \beta \in A_j \setminus \alpha \right)$ denote the remaining $r$ symbols of the recovering set.
    
    To find the value $c_{\alpha} = f_a(\alpha)$, find the unique polynomial $\delta(x)$ s.t.
    \begin{itemize}
        \item $\text{deg}(\delta(x)) \leq r$
        \item $\delta(\beta) = c_{\beta} \quad \forall \beta \in A_j \setminus \alpha$
    \end{itemize}
    
    This polynomial is:
    $$\delta(x) = \sum_{\beta \in A_j \setminus \alpha} c_{\beta} \prod_{\beta ' \in A_j \setminus \{\alpha, \beta\}} \frac{x - \beta'}{\beta - \beta '}$$

    Finally, set $c_{\alpha} = \delta(\alpha)$.
    
\end{frame}

\begin{frame}
    \begin{thm}
        The linear code $\C$ defined has dimension $k$ and is an optimal $(n,k,r)$ LRC code.
    \end{thm}
\end{frame}

\begin{frame}
    \begin{proof}[Proof of dimension]
        For $i \in \{0, \dots, r-1 \}$; $j \in \{0, \dots, \frac{k}{r-1}\}$ the $k$ polynomials $g(x)^j x^i$ all are of distinct degrees, i.e. linearly independent over $\FF$.
        
        $\Rightarrow$ The mapping $a \mapsto f_a$ is injective.
        
        $$\mbox{deg}(f_a(x)) \leq \mbox{deg}(x^{r-1}) + \mbox{deg}(g(x)^{\frac{k}{r}-1}) = r-1 + (r+1)(\frac{k}{r}-1)$$
        $$= k + \frac{k}{r} - 2 \leq n - 2$$
        
        This means that two distinct encoding polynomials give rise to two distinct codevectors. $\quad \Rightarrow \quad $ The dimension of the code is $k$.
    \end{proof}
\end{frame}

\begin{frame}
    \begin{proof}[Proof of optimality]
        Since the encoding is linear:
        $$d(\C) \geq n - \max_{f_a, a\in \FF_q^k} \mbox{deg}(f_a) = n - k - \frac{k}{r} + 2 \geq n - k - \left\lceil\frac{k}{r}\right\rceil + 2$$
        But we have that $d(\C) \leq n - k - \left\lceil\frac{k}{r}\right\rceil + 2$.
        Therefore, we have equality and thus it is an optimal LRC Code.
    \end{proof}
\end{frame}
    
    \begin{frame}{Example: (9,4,2) LRC code}
        We will now construct a $(n=9, k=4, r=2)$ LRC code over the field $\FF_q$.
        
        $$q = \vert \FF_q \vert \geq n \quad \Rightarrow \quad q \geq 9$$
        
        Choose $q = 13$
        
        $$\A = \{ A_1 = \{1,3,9\}, A_2 = \{2,6,5 \}, A_3 = \{4,12,10 \} \}$$.
        
        $$g(x) = x^3 = 
        \begin{cases}
            1 & \mbox{if } x \in A_1 \\
            8 & \mbox{if } x \in A_2 \\
            12& \mbox{if } x \in A_3
        \end{cases}
        $$
    \end{frame}
    
    \begin{frame}
        For $a = 
        \begin{pmatrix}
            a_{00} & a_{01} \\
            a_{10} & a_{11}
        \end{pmatrix} \in \FF_{13}^4$ define the encoding polynomial:
        $$f_a(x) =
        \begin{pmatrix}
            1 & x
        \end{pmatrix}
        \begin{pmatrix}
            a_{00} & a_{01} \\
            a_{10} & a_{11}
        \end{pmatrix}
        \begin{pmatrix}
            1 \\
            x^3
        \end{pmatrix} = a_{00} + a_{10}x + a_{01} x^3 + a_{11} x^4
        $$
        
        E.g. $a =         
        \begin{pmatrix}
            1 & 1 \\
            1 & 1
        \end{pmatrix}
        $. $f_a(x) = 1 + x + x^3 + x^4$
        
        $$c = (f_a(1), f_a(3), f_a(9), f_a(2), f_a(6), f_a(5), f_a(4), f_a(12), f_a(10))$$
        $$= (4,8,7,1,11,2,0,0,0)$$
    \end{frame}        
    
    \begin{frame}
        \onslide<1->{
        $$\delta(x) = \sum_{\beta \in A_j \setminus \alpha} c_{\beta} \prod_{\beta ' \in A_j \setminus \{\alpha, \beta\}} \frac{x - \beta'}{\beta - \beta '}$$    
    
        $$(
        \only<1,6->{f_a(1)}\only<2-5>{\textcolor{red}{\xcancel{f_a(1)}}},
        f_a(3),
        f_a(9),
        \only<1-5,10->{f_a(2)}\only<6-9>{\textcolor{red}{\xcancel{f_a(2)}}},
        f_a(6),
        f_a(5),
        \only<1-9>{f_a(4)}\only<10-13>{\textcolor{red}{\xcancel{f_a(4)}}},
        f_a(12),
        f_a(10))
        $$
        $$(
        \only<1,6->{4}\only<2-5>{\textcolor{red}{\xcancel{4}}},
        8,
        7,
        \only<1-5,10->{1}\only<6-9>{\textcolor{red}{\xcancel{1}}},
        11,
        2,
        \only<1-9>{0}\only<10-13>{\textcolor{red}{\xcancel{0}}},
        0,
        0
        )$$
        }
        \begin{overlayarea}{\textwidth}{0.3\textheight}
        \only<2-5>{
            \only<3-5>{$$ 1 \in A_1 = \{ 1, 3, 9 \}$$}
            \only<4-5>{$$ \Rightarrow \delta(x) = c_3 \frac{x-9}{3-9} + c_9 \frac{x-3}{9-3} = 2x + 2 $$}
            \only<5>{$$\delta(1) = 4$$}
        }
        
        \only<6-9>{
            \only<7-9>{$$ 2 \in A_2 = \{ 2, 6, 5 \}$$ }
            \only<8-9>{$$ \Rightarrow \delta(x) = c_6 \frac{x-5}{6-5} + c_5 \frac{x-6}{5-6} = 9x + 9$$}
            \only<9>{$$\delta(2) = 1$$}
        }
        
        \only<10-13>{
            \only<11-13>{$$4 \in A_3 = \{ 4,12,10 \}$$}
            \only<12-13>{$$ \Rightarrow \delta(x) = c_{12} \frac{x-10}{12-10} + c_{10} \frac{x-12}{10-12} = 0$$}
            \only<13>{$$\delta(4) = 0$$}
        }
        \end{overlayarea}
        
    \end{frame}
    
    \begin{frame}
        \frametitle{Example of LRC-2 code}
        
        Let $\FF = \FF_{13}$, $A = \FF \setminus \{0\}$
        
        $\mathcal{A} = \left\lbrace  \left\lbrace 1, 5, 12 , 8 \right\rbrace, \left\lbrace 2 , 10 , 11 , 3 \right\rbrace , \left\lbrace 4 , 7 , 9 , 6 \right\rbrace \right\rbrace$
        
        $\mathcal{A'} = \left\lbrace  \left\lbrace 1 , 3 , 9 \right\rbrace, \left\lbrace 2 , 6 , 5 \right\rbrace , \left\lbrace 4 , 12 , 10 \right\rbrace , \left\lbrace 7 , 8 , 11 \right\rbrace \right\rbrace$
        
        $f_a(x) = a_0 + a_1 x + a_2 x^4 + a_3 x^6$
        
        $a = (1,1,1,1)$

        $c = (4,8,7,5,2,6,2,2,2,3,9,1)$
        
        As already seen: $\delta(x) = 2x + 2$; $\delta(1)=4$.
        $$\delta ' (x) = c_5 \frac{x-12}{5-12}\frac{x-8}{5-8} + c_{12} \frac{x-5}{12-5}\frac{x-8}{12-8} + c_8 \frac{x-5}{8-5}\frac{x-12}{8-12}$$
        $$ = 6 \cdot 5 \cdot (x^2 + 6x + 5) + 2 \cdot 7 \cdot (x^2 + 1) + 9 \cdot 1 \cdot (x^2 + 9x + 8)$$
        $$ = x^2 + x + 2 \quad \longrightarrow \quad \delta ' (1) = 4$$
    \end{frame}        
    
    \begin{frame}
        Every coordinate $i$ has $t$ disjoint recovering sets $R_1^i, \dots, R_j^i$, each of size $r$, where $R_j^i \subset \left[n \right] \setminus i$.
        \begin{block}{Definition}
            The \textbf{recovering graph} of a $(n,k,r,t)$ LRC code $\mathcal{C}$ is an directed graph $G=(V,E)$ where:
            \begin{itemize}
                \item $V = \left[n \right]$. The set of vertices corresponds the set of $n$ coordinates of $\mathcal{C}$.
                \item $(i,j) \in E \iff j \in R_l^i$ for some $l \in \left[ t \right]$.\\
                There is an edge $i \rightarrow j$ if $j$ is in a recovering set of $i$.\\
                Note that $N(i) = \bigcup_{l = 1}^{t} R_l^i$
            \end{itemize}
        \end{block}
    \end{frame}    
    
    
\end{document}